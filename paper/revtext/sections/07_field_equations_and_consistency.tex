\subsection{Field equation}

Variation of the action~\eqref{eq:ActionPoisson} with respect to $\phi$ yields the
Euler--Lagrange equation
\begin{equation}
\boxed{
-\Delta_{G}\phi(\theta)
=
-\gamma\,A(\theta;q).
}
\label{eq:PoissonEquation}
\end{equation}

This equation is a Poisson equation on the Fisher--Rao manifold $(\ThetaMan,G)$,
with the empirical diagnostic acting as a geometric source.
Its structure is completely fixed by the Fisher geometry and the variational principle.

\subsection{Global consistency and zero-mode structure}

On compact statistical manifolds, the Laplace--Beltrami operator admits a constant
zero mode. As a direct consequence, Eq.~\eqref{eq:PoissonEquation} admits solutions
if and only if the empirical source satisfies the global consistency condition
\begin{equation}
\int_{\ThetaMan} \mathrm{d}^D\theta\,\sqrt{\det G}\,A(\theta;q) = 0.
\label{eq:zeromode}
\end{equation}

This condition is not an additional assumption but a structural consequence of the
Laplace--Beltrami operator on compact manifolds. Geometrically, it reflects the fact
that only the spatially varying component of empirical deformation can be relaxed
by Fisher--geometric smoothing. A uniform isotropic offset corresponds to a global
shift that cannot be resolved locally and must therefore be fixed by convention.

On non-compact statistical manifolds, the constraint~\eqref{eq:zeromode} is replaced
by appropriate decay conditions at large Fisher--Rao distance, which ensure both
existence and uniqueness of solutions as well as finiteness of the energy functional.
In practice, statistical manifolds of interest—such as exponential-family models
with open parameter domains—admit natural decay conditions inherited from the Fisher
volume measure.

In the univariate Gaussian example discussed in Sec.~\ref{sec:example}, the Poincaré
half-plane geometry guarantees sufficient decay of admissible solutions at infinity,
thereby ensuring a well-posed Poisson problem without the need for additional
infrared regularization.

\subsection{Gauge fixing and normalization}

The existence of a zero mode implies a gauge freedom under constant shifts
$\phi \mapsto \phi + \mathrm{const}$.
This freedom may be fixed by imposing a normalization condition, such as
\begin{equation}
\int_{\ThetaMan} \mathrm{d}^D\theta\,\sqrt{\det G}\,\phi(\theta) = 0.
\end{equation}

This choice fixes the reference level of the alignment field without affecting any
observable geometric gradients or relative structure.
