Empirical data generically deform the intrinsic Fisher--Rao geometry of 
statistical models, producing anisotropic score covariances and heterogeneous 
sensitivity profiles across parameter space. While such deformations can be 
quantified pointwise through invariant diagnostics constructed from the 
Fisher-normalized empirical covariance operator, static diagnostics alone
provide neither a variational principle nor a geometrically intrinsic mechanism
for regularization.

In this work we construct a minimal variational theory on statistical manifolds
by promoting the isotropic component of empirical alignment to an auxiliary scalar
field defined over parameter space and coupled to an observable, data-dependent
source. The resulting Fisher--geometric action is local, reparametrization invariant,
and introduces no structure beyond that induced by the Fisher--Rao metric itself.
Its Euler--Lagrange equation is a Poisson equation on the statistical manifold,
describing intrinsic geometric relaxation and smoothing of empirical deformation
measured in Fisher--Rao distance.

The theory is intentionally restricted to the scalar (spin-0) sector of empirical
geometric deformation. This choice isolates the leading-order isotropic component
of mismatch between model and data geometry, providing a principled and interpretable
geometric regularization framework while leaving anisotropic tensorial extensions as
a natural direction for future work. The \LaTeX{} source and all figures associated
with this work are publicly available at \url{https://github.com/isaidcornejo/coherence-field-action}.
