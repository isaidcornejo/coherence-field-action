The term ``dynamics'' is used here in a purely geometric sense. The coordinates
$\theta$ label statistically distinguishable models on the statistical manifold
$\ThetaMan$, not points in physical spacetime. Consequently, derivatives with respect
to $\theta$ describe variation across model space rather than temporal evolution.

Within this interpretation, the quadratic form
\begin{equation}
\Metric^{ij}\partial_i\phi\,\partial_j\phi
\end{equation}
penalizes sharp spatial variations of the alignment field between statistically
nearby models, as measured by Fisher--Rao distance. The Fisher geometry therefore
defines the intrinsic notion of locality and smoothness appropriate for empirical
deformation on parameter space.

\paragraph{Relation to regularization theory.}
Formally, the variational structure introduced here is related to manifold regularization
and Laplacian smoothing \cite{Belkin2006,CoifmanLafon2006}. The crucial distinction is
that the smoothing operator is not introduced algorithmically or heuristically. Instead,
it emerges uniquely as the Laplace--Beltrami operator associated with the Fisher--Rao metric,
derived from a variational principle constrained solely by reparametrization invariance, locality,
and the absence of additional geometric structure \cite{Rosenberg1997,Amari2016}.
