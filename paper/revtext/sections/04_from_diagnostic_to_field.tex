The central conceptual step of this work is to treat empirical geometric deformation
not merely as a pointwise diagnostic, but as a geometric field defined intrinsically
over the statistical manifold $\ThetaMan$.

While one could in principle apply ad hoc smoothing procedures directly to the scalar
diagnostic $A(\theta;q)$, such approaches lack a variational characterization and depend
on extrinsic regularization choices. In contrast, introducing an auxiliary field allows
the smoothing mechanism to be derived uniquely from the intrinsic Fisher--Rao geometry
of the statistical manifold.

We therefore introduce a scalar \emph{alignment field} $\phi(\theta)$, defined on $\ThetaMan$,
which represents a smooth geometric response to empirical deformation. The empirical diagnostic
$A(\theta;q)$ enters the theory as an externally observed, data-dependent source.

Throughout this work, the term ``field'' is used in the minimal geometric sense of an auxiliary
scalar function defined on a statistical manifold. No implication of physical spacetime dynamics,
propagating degrees of freedom, or temporal evolution is intended.

The field $\phi$ is not an observable quantity. Rather, it is an auxiliary geometric field whose
role is to encode a regularized, manifold-aware response to empirical mismatch. No \emph{a priori}
identification between $\phi$ and $A$ is imposed; their relationship emerges from a variational principle
intrinsic to Fisher geometry and fixed uniquely by reparametrization invariance and locality.

This construction is intentionally restricted to the scalar (spin-0) sector, which captures the
leading isotropic component of empirical deformation compatible with reparametrization invariance
and minimal geometric structure. Anisotropic, higher-rank tensorial extensions are left as a natural
direction for future work.
