The Fisher--Rao metric endows parametric statistical models with a canonical Riemannian
structure, encoding infinitesimal statistical distinguishability between nearby parameter
values \cite{Fisher1922,Rao1945,AmariNagaoka2000,Amari2016}. This intrinsic geometry
underlies asymptotic inference, natural-gradient methods \cite{Martens2020}, and modern
geometric formulations of learning and optimization \cite{Bronstein2021}.

Throughout this work, the statistical manifold is denoted by $\ThetaMan$, with local
coordinates $\theta \in \ThetaMan$ and dimension $D = \dim(\ThetaMan)$. While some
treatments distinguish between a parameter space and an abstract manifold, we adopt the
notation $\ThetaMan$ to emphasize the dual role of model parameters as coordinates on an
intrinsic information-geometric manifold endowed with a unique, reparametrization-invariant
metric structure.

In idealized settings where the data-generating distribution coincides with the assumed
model family, the Fisher--Rao metric fully characterizes local statistical sensitivity. In
empirical practice, however, the observed data distribution $q(x)$ generically deviates
from the assumed model $p(x \mid \theta)$. This mismatch induces anisotropic empirical
score covariances, dominant sensitivity directions, and effective dimensional reduction in
parameter space, phenomena that are well documented in modern statistical models and
learning systems \cite{Martens2020,Bronstein2021}. From an information-geometric
perspective, these effects correspond to a deformation of the intrinsic Fisher geometry
induced by empirical data.

Specifically, the empirical score covariance
\begin{equation}
\Cov_{ij}(\theta;q)
=
\mathbb{E}_{q}\!\left[
\partial_i \log p(x\mid\theta)\,
\partial_j \log p(x\mid\theta)
\right]
\end{equation}
does not, in general, coincide with the Fisher--Rao metric
\begin{equation}
\Metric_{ij}(\theta)
=
\mathbb{E}_{p(\cdot\mid\theta)}\!\left[
\partial_i \log p(x\mid\theta)\,
\partial_j \log p(x\mid\theta)
\right].
\end{equation}

Here $q(x)$ may denote either the true data-generating distribution or the empirical
measure induced by finite samples. Since only products of first-order score functions
appear, the empirical covariance remains well defined even when $q$ is atomic
\cite{AmariNagaoka2000,Amari2016}.

The discrepancy between $\Cov_{ij}$ and $\Metric_{ij}$ encodes how empirical data
selectively reinforce or suppress sensitivity directions relative to the model’s intrinsic
information geometry. Quantifying this deformation invariantly, and constructing a
principled mechanism to relax and geometrically smooth its empirical manifestations across
parameter space, constitute the central motivation of the present work.

The present work should be read as a geometric completion of the scalar diagnostic
introduced in Ref.~\cite{CornejoAlignment2025}. While the diagnostic $A(\theta;q)$ provides
a pointwise, reparametrization-invariant measure of empirical deformation, it does not
define a global or variational structure. Here we promote this diagnostic to a source for an
intrinsic geometric variational theory, yielding a globally consistent and Fisher--Rao--aware
regularization mechanism. The \LaTeX{} source of the manuscript and all figures are
publicly available at \url{https://github.com/isaidcornejo/coherence-field-action}.
