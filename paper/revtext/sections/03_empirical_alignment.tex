A coordinate-invariant measure of empirical geometric deformation is obtained by
forming the mixed-index alignment operator
\begin{equation}
\Align^{i}{}_{j}(\theta;q)
=
(\Metric^{-1})^{ik}(\theta)\,\Cov_{kj}(\theta;q),
\end{equation}
which compares empirical and model sensitivities in Fisher-normalized units.
By construction, $\Align^{i}{}_{j}$ transforms as a $(1,1)$ tensor under smooth
reparametrizations of $\ThetaMan$, and its spectrum provides a direction-wise
comparison between empirical and intrinsic Fisher geometry \cite{AmariNagaoka2000,Amari2016}.

From this operator one constructs the scalar diagnostic
\begin{equation}
A(\theta;q)
=
\Tr(\Align(\theta;q)) - D,
\end{equation}
where $D=\dim(\ThetaMan)$. This quantity vanishes at Fisher equilibrium and aggregates
reinforced and suppressed sensitivity directions into a single reparametrization-invariant
scalar observable. Importantly, $A(\theta;q)$ depends only on the trace of the alignment
operator and is therefore insensitive to the detailed directional structure of empirical
anisotropy \cite{CornejoAlignment2025}.

The diagnostic $A(\theta;q)$ isolates the isotropic component of empirical geometric
deformation, analogous to a geometric pressure acting uniformly on the statistical
manifold. In contrast, the traceless part of $\Align^{i}{}_{j}$ encodes anisotropic distortions
that may be interpreted as shear- or stress-like deformations of Fisher geometry, in direct analogy
with the scalar--tensor decomposition of geometric response fields \cite{Amari2016,Rosenberg1997}.
While these tensorial components carry additional directional information, they are not captured by
a single scalar invariant and would require a higher-rank geometric description.

The present work therefore focuses exclusively on the scalar (spin-0, pressure-like) sector
of empirical deformation. This choice yields the minimal and most robust invariant diagnostic
compatible with reparametrization invariance and locality, and provides a natural entry point
for a variational formulation without introducing additional geometric structure.

\begin{figure}[t]
\centering
\includegraphics[width=0.75\linewidth]{figures/fig_alignment_operator_spectrum.pdf}
\caption{Schematic spectrum of the alignment operator $\Align^{i}{}_{j}$. The trace captures
the isotropic (pressure-like) component of empirical deformation, while traceless anisotropic
modes correspond to shear-like distortions not addressed in the present scalar theory.}
\label{fig:alignment_spectrum}
\end{figure}
